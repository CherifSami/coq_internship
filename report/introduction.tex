\chapter{Introduction}
The following report describes the activities carried out during a 12-week, full-time internship at the Research Center in Computer Science, Signal and Automatic Control of Lille (CRIStAL). This internship revolves around the PIP pretokernel and the DEC language currently being developed by the 2XS team. PIP is an OS protokernel specified in Coq in terms of a shallow embedding of a fragment of the C language on which memory isolation is being proved using Hoare logic. DEC is a language, deeply embedded in Coq, designed as an intermediate language for the translation of PIP to C. Specifying programs in the deep embedding has the great advantage of simplifying their syntactic manipulation. It also ensures a stricter structuring of program expressions. \textbf{Therefore, we want to see whether this structure is reflected in the proofs and compare such proofs to those already given in the shallow embedding.}\\ 

To that end, we will consider three distinct invariants in the shallow embedding. For each of them, we will model the program function in the deep embedding and carry out the proof for its corresponding invariant. The first invariant function reads the memory. The second one writes in the memory. The last one is a recursive function. One of the invariants propagates all of PIP's properties which include memory isolation, vertical sharing, kernel data isolation and consistency. We followed a modular approach in our implementation and we engineered our proofs correspondingly. We also did some preliminary work mostly to get familiar with Hoare logic and the deep embedding. \\

The first part of the report offers an overview of the CRIStAL laboratory, the 2XS team and their research activities. The second part is dedicated to the PIP protokenerel and focuses on its proof oriented design, its properties, its data structures, Hoare logic theory, the deep embedding and its constructs. The last part is dedicated to our contributions, detailing how we modelled the required functions, how we engineered our proofs in the deep embedding and our observations about the comparison between the deep and shallow proofs.
  



