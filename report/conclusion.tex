\chapter{Conclusion}
In this report, we explained how we proved three different invariants of PIP in the deep embedding after modelling their corresponding functions while following a modular approach. The results we obtained were interesting with respect to our initial conjecture concerning proof structuring. Indeed, although the deep embedding ensures a stricter structuring of program expressions, this structure isn't reflected in naive proofs done by inversion. However, by using Hoare triple rules we managed to make our proofs more structured than shallow ones by decomposing our programs to simple instructions instead of considering them as a whole. Nevertheless, specifications of such proofs remain more complicated than the ones done in the shallow embedding. \textit{Wildmoser} and \textit{Nipkow} identified the same disadvantage while working on a deep assembly language using the Isabelle proof assistant\cite{Wildmoser}. Aside from the main differences between deep and shallow proofs, we also pointed out the limitations of the deep embedding and mainly error handling and pattern matching. \\

Many different adaptations, tests, and experiments have been left for the future due to lack of time. Possible future work may also involve the automation of proofs in the deep embedding as they seemed monotonous when done by inversion and predictable when it came to applying Hoare rules as each one was devised for a specific instruction of DEC. This is also possible since we are dealing with a closed language. \\

Logic theory and formal proofs in particular remain two of the most currently researched scientific fields. Indeed, proving that programs are correct became essential especially when we are dealing with critical systems. It is also important to make these proofs as structured as possible in order to make them legible and amendable. This is also crucial if we want to simplify further these proofs.  \\   

This internship was a beneficial first-hand research experience in such a renowned laboratory as CRIStAL. And although many technical difficulties were encountered, overcoming those challenges was the best way to learn. This experience was not only  constructive but also enjoyable as the atmosphere within the team was both professional and friendly. \\ 