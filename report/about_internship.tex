\chapter{About the internship}

\section{Objectives}

\section{Coq \& Coqide}
\begin{wrapfigure}[6]{r}{0.2\textwidth}
\vspace{-14pt}
\centering
\includegraphics[frame]{img/CoqLogo.png}
\end{wrapfigure}
Coq is the result of about 30 years of research.
It started in 1984 as an implementation of the Calculus of Constructions, an expressive formal language, at INRIA by Thierry Coquand and Gérard Huet and was extended,later in 1991, by Christine to the Calculus of Inductive Constructions. \\

Coq is a \textbf{formal proof management system}. It provides a formal language to write mathematical definitions, executable algorithms and theorems together with an environment for semi-interactive development of machine-checked proofs. Typical applications include the certification of properties of programming languages (e.g. the CompCert compiler certification project, or the Bedrock verified low-level programming library), the formalization of mathematics (e.g. the full formalization of the Feit-Thompson theorem or homotopy type theory) and teaching. It implements a program specification and mathematical higher-level language called Gallina that is based on the Calculus of Inductive Constructions combining both a higher-order logic and a richly-typed functional programming language.\\

As a \textbf{proof development system}, Coq provides interactive proof methods, decision and semi-decision algorithms as well as a tactic language letting the user define his own proof methods. Furthermore, as a \textbf{platform for the formalization of mathematics or the development of programs}, Coq provides support for high-level notations, implicit contents and various other useful kinds of macros.\\

For this internship, the latest version of Coq, 8.6 released in December 2016, was used. It features among other things a faster universe checker, asynchronous error processing, proof search improvements, generalized intro patterns, a new warning system, patterns in abstractions and a new subterm selection algorithm.

%logo

\section{organization}
